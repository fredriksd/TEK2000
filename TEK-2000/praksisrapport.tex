\documentclass[11pt, a4paper]{report}

\usepackage[a4paper,width=150mm,top=25mm,bottom=25mm]{geometry}
\usepackage[T1]{fontenc} 								
\usepackage[norsk]{babel}								
\usepackage[utf8]{inputenc}						
\usepackage{graphicx}       						
%\usepackage{amsmath,amssymb}
%\usepackage{grffile}
%\usepackage{listings}
%\usepackage{caption}
%\usepackage[export]{adjustbox}
\usepackage{titling}
%\setcounter{secnumdepth}{0}

%\lstset{extendedchars=true, basicstyle=\footnotesize, numbers=left, numberstyle=\tiny, frame=shadowbox, tabsize=2, language=C, showstringspaces=false, breaklines=true,
%  literate=
  %{á}{{\'a}}1 {é}{{\'e}}1 {í}{{\'i}}1 {ó}{{\'o}}1 {ú}{{\'u}}1
  %{Á}{{\'A}}1 {É}{{\'E}}1 {Í}{{\'I}}1 {Ó}{{\'O}}1 {Ú}{{\'U}}1
  %{à}{{\`a}}1 {è}{{\`e}}1 {ì}{{\`i}}1 {ò}{{\`o}}1 {ù}{{\`u}}1
 % {À}{{\`A}}1 {È}{{\'E}}1 {Ì}{{\`I}}1 {Ò}{{\`O}}1 {Ù}{{\`U}}1
  %{ä}{{\"a}}1 {ë}{{\"e}}1 {ï}{{\"i}}1 {ö}{{\"o}}1 {ü}{{\"u}}1
  %{Ä}{{\"A}}1 {Ë}{{\"E}}1 {Ï}{{\"I}}1 {Ö}{{\"O}}1 {Ü}{{\"U}}1
  %{â}{{\^a}}1 {ê}{{\^e}}1 {î}{{\^i}}1 {ô}{{\^o}}1 {û}{{\^u}}1
  %{Â}{{\^A}}1 {Ê}{{\^E}}1 {Î}{{\^I}}1 {Ô}{{\^O}}1 {Û}{{\^U}}1
  %{ã}{{\~a}}1 {Ã}{{\~A}}1 {õ}{{\~o}}1 {Õ}{{\~O}}1
  %{œ}{{\oe}}1 {Œ}{{\OE}}1 {æ}{{\ae}}1 {Æ}{{\AE}}1 {ß}{{\ss}}1
  %{ç}{{\c c}}1 {Ç}{{\c C}}1 {ø}{{\o}}1 {å}{{\r a}}1 {Å}{{\r A}}1
  %{€}{{\EUR}}1 {£}{{\pounds}}1}

\setlength{\textheight}{240mm} 
\setlength{\textwidth}{180mm}  
\topmargin -5mm 
\oddsidemargin -5mm

\pretitle{%
  \begin{center}
  \LARGE
  \includegraphics[width=6cm,height=6cm]{uitlogo.png}
}
\posttitle{\end{center}}

\begin{document}
\title{\\Praksis hos Norut}
\author{Fredrik Sandhei\thanks{UiT - TEK-2000, obligatorisk rapport.}}

\date{\today}
\maketitle
\newpage
\tableofcontents
\newpage

\chapter{Introduksjon}
%Beskriv emnet, hensikten og hva som må gjøres
%FÅ MAL FRA EMNEANSVARLIG
\chapter{Hendelsesforløp}
%Kort om hver uke. Få frem meningen med alt
%Få beskrevet arbeidsoppgavene, forhåpentligvis ei liste 
%som blir oversiktlig.
%Litt frem og tilbake på hva jeg driver med
\section{Uke 1}
Den første dagen begynte med introduksjon av meg til personalet og bli kjent med lokale. Deretter ble jeg satt rett i arbeid med backlog av tidligere flight plan inn i det GIS-baserte loggsystemet. \\
Typisk metodikk innen luftfart (fra erfaringsmessig område) er "learning by doing". Poenget er for best læringskapasitet med å bare "prøve ut" litt for seg selv, og ta utgangspunkt i eksempler som er gitt.\\
En flight log er et skjema som beskriver en flyoperasjon %kom med beskrivelse...

%Bilde av loggskjema
\begin{figure}[h]
	\caption{Typisk opplegg av loggskjema}
	\centering
	\includegraphics[height= 10cm, width=0.5\textwidth]{flightlogNorut.png}
\end{figure}

Mesteparten av den første dagen gikk til backlog samt en avdelingsmøte som jeg fikk muligheten til å delta på (i sammenheng til å bli introdusert).
%Back log av flight plans
%Flyoperativ avdeling 
%Avdelingsmøte
%Teknisk avdeling
\section{Uke 2}
\section{Uke 3}
\section{Uke 4}

\chapter{Diskusjon \& refleksjon}
\section{Lærdom}


\section{Konklusjon}



\end{document}